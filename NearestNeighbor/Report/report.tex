\documentclass[twoside,11pt]{article}

\usepackage{jmlr2e}
\usepackage{spverbatim}

\begin{document}

\title{Learning with Nearest Neighbor}

\author{\name Andrew Kirby \email andy-kirby@live.com \AND
		\name Kevin Browder \email browderkevin54@gmail.com \AND
		\name Nathan Stouffer \email nathanstouffer1999@gmail.com \AND
		\name Eric Kempf \email erickempf123@gmail.com }

\maketitle

\begin{abstract}
	blah Abstract deez nuutz
	
	we learned the relative performances of these algs
	knn generally better, tradeoff performance for less data storage with other algs
	
	Andy 
	
\end{abstract}

\section{Problem Statement w/ Hypothesis}

In the realm of machine learning, knowing the form of some distribution of data is very valuable. In these cases, parametric learning my be used, mimicking the distribution with a fixed set of parameters and forms. An example of this is Naive Bayes, which can be learned quickly and represented simply. These learning algorithms train fast and require relatively less data. The limitation of parametric learning is that the functional form of the algorithm constrains the performance.

The data sets to be handled in this project do not follow a known distribution. Thus, constructing parametric learners for these sets is very hard. Non-parametric learning offers a solution to this problem, making no strong assumptions about the form of the data---and therefore not constructing a model with a fixed form. 

One popular and intuitive method of non-parametric learning is K-Nearest Neighbors (K-NN) learning algorithm. Given data sets containing examples, the task is to implement K-NN learning algorithm for both classification and regression problems. Given the same input file, a secondary task is to reduce the size of the data set by implementing the Edited Nearest Neighbor (E-NN), Condensed Nearest Neighbor (C-NN), K-Means (KMeans), and K-Medoids (PAM) data reducing algorithms. Of particular are interest is how the data reducers contribute to the performance of K-Nearest Neighbor.

\subsection*{Hypothesis}

K-NN is expected to outperform the other nearest neighbor variants. K-NN utilizes the most data points from the data set, which is essential for minimizing the error rate. Furthermore, KMeans and PAM are expected to perform better than E-NN and C-NN because they will have similar amounts of representative points, but KMeans and PAM fit the representative points to the known data.

\section{Algorithms}

\subsection{K-NN}

K Nearest Neighbor (K-NN) is a modification of the nearest neighbor rule, a nonparametric procedure for classification or regression that utilizes a set of known points.

At its core, the nearest neighbor rule utilizes the intuitive assumption that a set of known examples, $X$, follows the distribution of all possible examples that it was taken from. As such, examples which are ``close" to each other likely belong to the same class or have a similar real value (\cite{NNClassification}). Metrics for determining this distance are discussed in section 2.2.

Thus, given $X$, a number of neighbors $k$, and a new example $x$, K-NN finds the $k$ nearest neighbors of $x$ by iterating through the examples in $X$, calculating the distance from $x$ for each example. Then K-NN classifies $x$ given a majority vote among the returned class values from its $k$ nearest neighbors. K-NN handles regression by computing the mean value among its $k$ nearest neighbors.

Note that this does not heavily assume anything about the form of the actual distribution. K-NN is flexible enough to adapt to almost any functional form that the underlying distribution takes on. This is a huge advantage for data sets that are hard to visualize or have unknown traits. The downside to this characteristic is that, like many nonparametric models, K-NN requires much more data than a parametric model to train effectively. In other words, K-NN requires a great number of known examples because it must is essentially estimating the real distribution from the known examples. Without sufficient data, K-NN will perform poorly.

In fact, the error of K-NN, $R_{knn}$ as the number of points sampled approaches infinity is bounded by:
$$R^* \leq R_{knn} \leq (1+\frac{1}{k})R^*$$ 
where $R^*$ is the Bayes risk, the optimal risk given a known set of probability densities that represents the data (\cite{NNClassification}). Thus, as $k \rightarrow \infty$ and the number of data points $n \rightarrow \infty$, theoretically $R_{knn} \rightarrow R^*$. Of course, $\frac{k}{n} \rightarrow 0$ should be ensured so that $k$ is still only a small subset of the total data points $n$. In short, K-NN will work optimally as the number of points sampled approaches infinity. As more attributes or data complexity is added, K-NN will approach its optimal condition slower given increasing $n$.


\subsection{Distance Metrics}

The nearest neighbor rule, and therefore K-NN, requires a distance metric to determine the theoretical ``distance" between two examples. A common distance metric---and the one used in this paper---is Euclidean distance. The Euclidean distance $D$ between two examples $x_1$ and $x_2$ is computed as:
$$D = \sqrt{\sum_{i=0}^{d}(a_1^i - a_2^i)^2}$$
where $d$ is the number of attributes the examples have, $a_1^i$ is the $i$-th attribute of $x_1$, and $a_2^i$ is the $i$-th attribute of $x_2$.

\subsection{Data Reducers}
The K-Nearest Neighbor decision rule is a lazy algorithm where the performance is a function of the number of training examples. Therefore, in terms of a learner's computational efficiency, the most favorable training set is the smallest set.
This is subject to the constraint that the model's performance is not compromised.

Thus, the optimal training set for a K-NN learner is the smallest set that is able to maximize the correct identification of any query example.
The purpose of the following four algorithms is to reduce the size of the dataset without compromising the learner's performance. Since K-NN is lazy by nature, this will improve the computation time of the learner at query time.
% should I talk about not decreasing performance? That is what we are testing right


\subsubsection{Edited K-NN}
The discussion of algorithms that reduce the size of a training set begins with Edited K-Nearest Neighbor. Edited K-NN begins by training a K-NN learner $L$ with each example in an initial dataset $D^0$. Note that since multiple steps in the following algorithm rely on knowing the classification of $e$, Edited K-NN can only be used on classification problems.

The learner $L$ then classifies each example $e^0 \in D^0$. That is, for each example $e^0$, the learner $L$ returns the predicted class of $e^0$: $predclass(e^0)$. However, recall that $e^0$ is a training example, so $class(e^0)$, the actual class of $e^0$, is known. So, for each $e^0$ where $class(e^0) \neq predclass(e^0)$ (if $e^0$ was misclassified), the algorithm removes $e^0$ from $D^0$.

This creates a new set $D^1$. The learner $L$ is now trained with $D^1$ and the editing process repeats itself until performance decreases. In this case, the performance is measured as the accuracy of the learner on a validation set. The validation set is consistent in all editing iterations.

Say performance decreases on the $i^{th}$ iteration. The final edited data set is $D^{i-1}$. A separate K-NN classifier is now trained using $D^{i-1}$ and is tested using a testing set.


% TODO talk about the k used in edited k-nn
% TODO talk about accuracy being our metric

% explain edited (cite paper)
% metric for decisions (accuracy)

% notes: edited paper has a good line of reasoning for the hypothesis
% nn-d2l has information for tuning
% remember to discuss lack of knowledge about overfitting in convergence based algorithms
% TODO get rid of K-NN in reducing algorithm titles

\subsubsection{Condensed K-NN}

The condensing algorithm is used to reduce total number of points in a data set. It does this by condensing smiliarly classed points that neighbor each other into a single point.

Condensing begins with a null set $Z$ and initializes a random starting point out of the data set. $Z$ becomes the reduced data set once the algorithm has completed. The algorithm then iterates through all examples $x$ in the original data set and finds the closest example $x’$ in $Z$ for each of them. It then compares the classes of $x$ and $x’$, adding $x$ to the data set $Z$ only if the two classes differ. This process is repeated until $Z$ no longer changes, or the max number of iterations is reached. 

This process can be thought of as simply taking all of the same class examples nearest each other and condensing them into a single point. This way, we reduce the number of points required to represent a given class at any given point. (\cite{Condensed})

\subsubsection{Kmeans K-NN}

The K-Means (C-Means) algorithm is a data reducer that utilizes clustering to return a smaller data set. It does this by taking the average value of all the points in a cluster, and updating them until the averages remain the same from iteration to iteration.

 The algorithm is passed in a data set and a value for number of clusters $K$. It then initializes $K$ clusters by randomly selecting points out of the data set. We chose to take the initial points from the data set in an effort to reduce the time it takes for the algorithm to zero in on the data. From there, the algorithm iterates through all of the examples in the original set and adds them to the nearest cluster. Once all points have been assigned, K-Means computes the new average point within each cluster and assigns that to the representative point. Once every cluster average has been recomputed, all points are reassigned to the new representative points. This process is repeated until the representative points no longer change, or a max number of iterations is reached.

By recalculating the means of each cluster multiple times, the algorithm is very predictable and the centroids will normally find there way to the same points, regardless of which points were first initialized. (\cite{CMeans})

\subsubsection{PAM K-NN}
The Partitioning Around Medoids (PAM) K-NN is a variant of K-Means K-NN but is more resistant to outliers because it uses an actual point in the data set instead of making a new point using means. PAM begins by randomly selecting $C$ number of data points from the data set $D$ as the medoids $m$. In this case $C$ is determined by the number of points returned by Edited K-NN. The algorithm then clusters all of the remaining examples $x$ to the closest corresponding $m_i \in m$. Then the distortion is calculated with $$D = \sum_{j=1}^{k}\sum_{i \in cluster_j} (x_i - m_j)$$. We then iterate through each $m$ and swap it with every $x_j \in D$ $x_j \notin m$. After every swap the distortion $D$ is calculated again. This was optimized by calculating the distortion of each medoid separately and only recalculating the distortion for the medoid that was changed. If the new distortion is less than the original distortion the algorithm moves on and if the new distortion is greater than the original distortion than the medoid and example are swapped back. The clusters are then cleared and the algorithm then recalculates all of the clusters and distortion and repeats the swapping until no more swapping occurs or the algorithm iterates through 1000 times (\cite{Fox1990FindingGI}).

\section{Experiment}

\subsection{Preprocessing Choices}

First all the examples in the data set are randomly scrambled and then assigned to sets for ten-fold cross validation. All categorical variables are converted to integers and the preprocessor also generates a similarity matrix for each categorical variable that is used for determining distanced between categorical variables. Class names are also converted to incrementing integers for easier reading by the Java program. All numerical variables are normalized between 0 and 1. The data sets did not contain any missing variables so preprocessing did not need to handle that. 

\subsection{Tuning}

The most pertinent parameter to tune is $K$ which is the number of nearest neighbors to compare to. The value starts with $K = sqrt(N)$ where $N$ is the number of examples in the data set. 

The second parameter to tune is $C$ which is the number of representative points that K-Means K-NN and K-Medoids K-NN have. Tuning begins with the number of points returned by Edited K-NN and tuned from there. 

\section{Results}

a few of us

A FULL PAGE (OR MORE) WITH GRAPHS OF EVAL METRICS for each data set. CORRESPONDING GRAPHS SIDE BY SIDE WITH K varied. SIDE BY SIDE WITH C varied.

if kevin didnt mention normalization in preprocessing, we can talk about it here?

\section{Summary}



\bibliography{report}

\end{document}