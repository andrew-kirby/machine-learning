\documentclass[twoside,11pt]{article}
\usepackage{jmlr2e}

\begin{document}

\title{Learning with Naive Bayes}

\author{\name Andrew Kirby \email andy-kirby@live.com \AND
		\name Kevin Browder \email browderkevin54@gmail.com \AND
		\name Nathan Stouffer \email nathanstouffer1999@gmail.com \AND
		\name Eric Kempf \email erickempf123@gmail.com }
	
\maketitle

\begin{abstract}
	
\end{abstract}

\section{Introduction}

Classification is an important aspect of machine learning. A classification algorithm takes a given set of attributes, called examples, and places them in a class. One popular classification algorithm is Naive Bayes. Of particular interest is the process of training the Naive Bayes classifier on various data sets and the key factors of its performance.

\section{Problem Statement}

The problem is a classification problem. Given an input file that contains examples (each example consists of a list of attributes and an associated classification), the task is to implement a learning algorithm that is trained to classify examples.

\subsection{Variables}

The independent variable is whether the data is scrambled or not. Scrambling is described as follows. First, $10\%$ of the attributes in a given data set are randomly selected. Then, within each attribute, the values are randomly swapped between examples. Now the data set is scrambled. The dependent variable is how well the algorithm performs.

\subsection{Hypothesis}

The hypothesis is that scrambling a given data set will not significantly change the performance of the Naive Bayes Algorithm. In essence, scrambling renders $10\%$ of attributes useless. Any pattern that existed before scrambling is no longer discernible within those attributes. However, there remains $90\%$ of the data that persists with the original pattern. Naive Bayes chooses a classification based on the relative probability that a given example is in a class. The order of the relative probabilities will vary little even when the original pattern is lost in $10\%$ of the attributes, yielding similar performance.

\section{Naive Bayes}

Naive Bayes is a low cost classifying algorithm that utilizes a probabilistic model based in Bayesian Decision Theory. Given an example $x \in X$ with attributes $a_1, a_2, ..., a_d$ and belonging to class $c \in C$, the algorithm will classify $x$ by choosing the maximum probability $P(c|a_1, a_2, ..., a_d)$ from all $c \in C$ \citep{nbPaper}.

Calculating this probability becomes more apparent when rewritten using Bayes Theorem:
$$P(c|a_1, a_2, ..., a_d) = \frac{P(a_1, a_2, ..., a_d | c)P(c)}{P(a_1, a_2, ..., a_d)} $$

Because the algorithm chooses a classification $c = argmax_{c \in C}P(c|a_1, a_2, ..., a_d)$, the classification is equivalent to
$$c = argmax_{c \in C}P(a_1, a_2, ..., a_d | c)P(c)$$

Naive Bayes assumes all attributes $a \in A$ are conditionally independent given the class, simplifying the classification decision to 
$$c = argmax_{c \in C}P(c)\prod_{i = 1}^{d}P(a_i | c)$$

These probabilities are calculated when the algorithm is trained with a given set of examples $X_{test} \subset X$. During training, $P(c)$, is calculated for all classes by
$$P(c) = \frac{|X_c|}{|X_{test}|}$$
where $X_c$ is the set of all examples that belong to the class $c$ and $P(a_i | c)$ is calculated with
$$P(a_i | c) = \frac{|X_{c_{a_i}}| + 1}{|X_c| + d}$$
where $X_{c_{a_i}}$ is the set of all examples in class $c$ which have attribute $a_i$ and $d$ is the number of attributes.

Once the algorithm has been trained, all probabilities $P(c)$ and $P(a_i | c)$ have been calculated and stored for future reference. Classification of a new example will be selected quickly with little further computation.

\section{Experimental Design}

The implementation of Naive Bayes will be trained and tested using ten-fold cross validation. First, the original data sets will be used for the cross validation, then 10\% of their attributes are shuffled (to introduce noise) and put through cross validation. The performance of the learning algorithm will be evaluated by two loss function metrics, accuracy and mean squared error. 


\subsection{Pre-processing}

All pre-processing is done using the pandas library in Python. The class column is moved before all of the attributes for a consistent output between datasets. Next the examples are randomly shuffled. A new column is added after the class column and each example is assigned to one of 10 sets that are used in the 10 fold cross validation. If the data is continuous, it is discretized into a chosen number of bins with each bin containing an equal number of examples. Non numerical data is changed to integers for ease of use with the algorithm. The data sets used do not have any missing values, so the preprocessing did not need to handle this \citep{datasets}.

Set information is included in the first three lines of the processed data. The first line includes the number of classes, number of attributes and the number of examples. The second line includes the number of bins for each attribute. The third line includes the class names so the algorithm can convert back from numerical class names to the original string names. After these lines are generated, they are outputted to the first three lines of the .csv file and the pre-processed data is outputted starting on the fourth line. After this data is outputted, ten percent of the attributes are randomly chosen and the data in each of these attributes is randomly shuffled. This shuffled data is then outputted to a new .csv with the same header as the first file.  

\subsection{Tuning}

The most pertinent parameter to tune is the number of bins used when discretizing an attribute with continuous output.
Tuning is done by changing the number of bins and then evaluating the loss functions for the different values. \\\\
As seen in Table \ref{tab:metrics2}, for 2 bins, the Accuracy ranges from $69.11\%$ to $91.92\%$ and the Mean Squared Error ranges from $0.65$ to $7.8$. This begins process of testing different numbers of bins to produce the best lost function.
\begin{table}[h]
	\centering
	\caption{Loss Function Metrics for num\_bins = 2} \label{tab:metrics2}
	\begin{tabular}{|l|l|l|}
		\hline
		Dataset                  & Accuracy & MSE  \\ \hline
		glass                    & 69.11\%  & 5    \\ \hline
		iris                     & 70.67\%  & 7.8  \\ \hline
		house-votes-84           & 89.66\%  & 4.3  \\ \hline
		soybean-small            & 79\%     & 0.65 \\ \hline
		wdbc                     & 91.92\%  & 7.4  \\ \hline
	\end{tabular}
\end{table} \\\\
In Table \ref{tab:metrics5}, where there are 5 bins, the Accuracy ranges from $79.50\%$ to $94.02\%$ and the Mean Squared Error ranges from $0.73$ to $6.1$.
\begin{table}[h]
	\centering
	\caption{Loss Function Metrics for num\_bins = 5} \label{tab:metrics5}
	\begin{tabular}{|l|l|l|}
		\hline
		Dataset                  & Accuracy & MSE  \\ \hline
		glass                    & 80.78\%  & 2.9  \\ \hline
		iris                     & 92.67\%  & 0.73 \\ \hline
		house-votes-84           & 90.13\%  & 6.1  \\ \hline
		soybean-small            & 79.50\%  & 0.9  \\ \hline
		wdbc                     & 94.02\%  & 4    \\ \hline
	\end{tabular}
\end{table} \\\\
With 5 bins, the minimum bound for accuracy increases by $10.49\%$ and the range of the Mean Squared Error slightly decreases.
This means that 5 bins performs better than 2 bins. \\\\
More testing shows that as the number of bins increases past 5, the performance of Naive Bayes stays consistent. The number of bins where the performance begins to drop is not known. \\\\
However, viewing Table \ref{tab:metrics1000}, it is clear that 1000 bins does not produce a good model for all data sets.
The Accuracy ranges from $5.13\%$ to $92.67$ and the Mean Squared Error ranges from $0.6$ to $1191.4$.
\begin{table}[h]
	\centering
	\caption{Loss Function Metrics for num\_bins = 1000} \label{tab:metrics1000}
	\begin{tabular}{|l|l|l|}
		\hline
		Dataset                  & Accuracy & MSE    \\ \hline
		glass                    & 5.13\%   & 80.67  \\ \hline
		iris                     & 92.67\%  & 0.87   \\ \hline
		house-votes-84           & 89.66\%  & 7.9    \\ \hline
		soybean-small            & 83.50\%  & 0.6    \\ \hline
		wdbc                     & 39.54\%  & 1191.4 \\ \hline
	\end{tabular}
\end{table} \\\\
Thus, the minimum number of bins required to produce a satisfactory model is 5.

\section{Results}

The results indicate that the hypothesis that scrambling a given data set will not significantly change the performance of the Naive Bayes Algorithm is true.
The results of the experiment are displayed in Table \ref{tab:metrics}.

\begin{table}[h]
	\centering
	\caption{Loss Function Metrics} \label{tab:metrics}
	\begin{tabular}{|l|l|l|}
		\hline
		Dataset                  & Accuracy & MSE  \\ \hline
		glass                    & 80.78\%  & 2.9  \\ \hline
		glass-scrambled          & 78.03\%  & 3.17 \\ \hline
		iris                     & 92.67\%  & 0.73 \\ \hline
		iris-scrambled           & 90.67\%  & 1    \\ \hline
		house-votes-84           & 90.13\%  & 6.1  \\ \hline
		house-votes-84-scrambled & 89.90\%  & 4.2  \\ \hline
		soybean-small            & 79.50\%  & 0.9  \\ \hline
		soybean-small-scrambled  & 79.50\%  & 0.9  \\ \hline
		wdbc                     & 94.02\%  & 4    \\ \hline
		wdbc-scrambled           & 93.85\%  & 3.3  \\ \hline
	\end{tabular}
\end{table}

\section{Summary}

For its relative cost and simplicity, Naive Bayes is a powerful model for classification. A surprising level of performance was noted given the small size of the data sets used. Although considered a more primitive algorithm, Naive Bayes may be further enhanced with appropriate smoothing techniques and tuning parameters. Knowing the characteristics of the data sets, including the level of attribute conditional independence and relevance, allows for further optimization. 

Drawbacks of Naive Bayes include the costliness of optimizing tuning parameters such as discrete bin size and attribute weights (if implemented). The model must be trained and tested in order to compare these parameters. Additionally, the algorithm is only designed to represent examples with conditionally independent attributes, which requires careful planning when deciding included attributes.


\bibliography{nbBib}

\end{document}