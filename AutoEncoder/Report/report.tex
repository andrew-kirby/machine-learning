\documentclass[twoside,11pt]{article}
%\documentclass[twoside,10pt]{article}

\usepackage{jmlr2e}
\usepackage{spverbatim}
\usepackage{amsmath}
\usepackage{caption}

\newcommand{\ith}{$i^{th}$ }
\newcommand{\jth}{$j^{th}$ }
\newcommand{\weight}{$w_{ji}$ }
\newcommand{\Rw}{\mathbb{R}^w }
\newcommand{\R}{\mathbb{R}}

\begin{document}

\title{Auto-Encoders for Feature Selection}

\author{\name Andrew Kirby \email andy-kirby@live.com \AND
		\name Kevin Browder \email browderkevin54@gmail.com \AND
		\name Nathan Stouffer \email nathanstouffer1999@gmail.com \AND
		\name Eric Kempf \email erickempf123@gmail.com }

\maketitle

\begin{abstract}

\end{abstract}

\section{Problem Statement}
	Each training algorithm will be implemented and tested on the following six data sets.
		
	The abalone dataset contains features paired with the age of an abalone (the class). The 28 ages are closely related, potentially making classification difficult.
	The car dataset contains attributes corresponding to 4 possible conditions of a car while the segmentation dataset details 6 distinct subjects of outdoor photographs.
	The rest of the datasets are regression.
	The forest fires set contains the area burned by a forest fire. Most of the values are 0 with a few large values which might give the model trouble.
	The machine dataset predicts the performance of a CPU which has an even distribution across its regression values. 
	Wine quality contains the score of a wine with associated attributes. Several of the attributes are correlated, making regression more difficult \citep{datasets}.

\subsection{Hypothesis}

\section{Algorithms}
		
\section{Experiment}

\subsection{Preprocessing Choices}

	First, all the examples in the data set are randomly scrambled and then assigned to sets for ten-fold cross validation. 
	All categorical variables are converted to integers 
	and the preprocessor also generates a similarity matrix for each categorical variable that is used for determining distances between categorical variables. 
	All numerical variables are normalized between 0 and 1. 
	The data sets did not contain any missing variables.

\subsection{Algorithm Choices}
	

\subsection{Evaluation Metrics}

	Classification datasets are evaluated with accuracy and mean squared error (MSE). 
	The MSE metric implemented measures the squared error between the predicted and actual class distributions. 
	This is similar in concept to a Brier score but slightly different in computation. Accuracy indicates how well the algorithm individually classifies examples.
	
	To evaluate regression datasets, mean error (ME) and MSE will be used. MSE squares the distance between a real and predicted value, the squares are then averaged over the entire testing set. 
	ME is computed similarly, but will not square the difference. MSE emphasizes the effect of outliers while ME captures whether the learner tends to over or underestimate the values in the test set. 
	MSE and ME are computed using z-scores (the number of standard deviations from the mean) so that comparisons can be made between datasets.

\subsection{Tuning}

\section{Results}

\section{Summary}

\newpage

\bibliography{biblio}

\end{document}
